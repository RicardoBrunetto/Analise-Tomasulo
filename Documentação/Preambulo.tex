\documentclass[12pt]{article}

\usepackage{cmap} %impedir ligadura de fi, fl
\usepackage[T1]{fontenc} % reconhecer acentos
\usepackage[ansinew]{inputenc}
\usepackage[brazil]{babel}
\usepackage{color}
\usepackage{graphics,graphicx}
\usepackage{subfigure} % pra inserir várias figuras no mesmo ambiente
\usepackage{longtable} % para inserir tabelas extensas sem serem quebradas
\usepackage{multirow} % pra permitir mescla de linhas na tabela
\usepackage{pdfpages}
\usepackage[titletoc,toc,page]{appendix}
\renewcommand{\appendixtocname}{Ap�ndices}
\renewcommand{\appendixpagename}{Ap�ndices}
%\usepackage{helvet} 			%fonte Helvetica
%\renewcommand{\familydefault}{\sfdefault}				%definir fonte sem serifa como
% padrão

%Referências
\usepackage[comma]{natbib}   % omit 'round' option if you prefer square brackets
\bibliographystyle{plainnat}
% - citet => Author (year)		|		- citeauthor => Author

% Margem e espa�amento
\usepackage[top=2.5cm, bottom=2.5cm, left=2.5cm, right=2.5cm]{geometry}
\renewcommand{\baselinestretch}{1.5}   % espa�amento entre linhas
\setlength{\parskip}{0.2\baselineskip} % espa�amento entre par�grafos
\usepackage{indentfirst} % indentar o primeiro par�grafo das se��es
\setlength{\parindent}{1.5cm} % tamanho da indenta��o de par�grafo
\usepackage{setspace} % permite comandos de espa�amento entre linhas

%Cabe�alho e rodap�
\usepackage{fancyhdr}
\pagestyle{fancy}
\renewcommand{\headrulewidth}{0pt}
\fancyhead[LE, LO, RE, RO]{}
\fancyfoot[CO, CE]{}
\fancyfoot[RO, RE]{\thepage}

%Novos Ambientes
\newenvironment{itens}{\begin{itemize}\nospaceitem}{\end{itemize}}
\newsavebox{\mybox}
\newenvironment{m_box}
{\begin{lrbox}{\mybox}\begin{minipage}{0.9\textwidth}}
{\end{minipage}\end{lrbox}\fbox{\usebox{\mybox}}}

\newcommand\notes[1]{%
\fbox{\begin{minipage}{0.9\textwidth}#1\end{minipage}}}
\newcommand{\minitem}[1]{\subitem {\small $\circ$ #1}}
\newcommand{\nospaceitem}{\setlength{\itemsep}{1pt}\setlength{\parskip}{0pt}\setlength{\parsep}{0pt}}
\newcommand{\pulalinha}[1]{\vspace{#1\baselineskip}}

% Para printar código em assembly MIPS32
\usepackage{color}

\definecolor{dkgreen}{rgb}{0,0.6,0}
\definecolor{gray}{rgb}{0.5,0.5,0.5}
\definecolor{mauve}{rgb}{0.58,0,0.82}


\usepackage{listings} % needed for the inclusion of source code
\usepackage{mips}

\lstset{ %
  language=[mips]Assembler,       % the language of the code
  basicstyle=\footnotesize,       % the size of the fonts that are used for the code
  numbers=left,                   % where to put the line-numbers
  numberstyle=\tiny\color{gray},  % the style that is used for the line-numbers
  stepnumber=1,                   % the step between two line-numbers. If it's 1, each line
                                  % will be numbered
  numbersep=5pt,                  % how far the line-numbers are from the code
  backgroundcolor=\color{white},  % choose the background color. You must add \usepackage{color}
  showspaces=false,               % show spaces adding particular underscores
  showstringspaces=false,         % underline spaces within strings
  showtabs=false,                 % show tabs within strings adding particular underscores
  frame=single,                   % adds a frame around the code
  rulecolor=\color{black},        % if not set, the frame-color may be changed on line-breaks within not-black text (e.g. commens (green here))
  tabsize=4,                      % sets default tabsize to 2 spaces
  captionpos=b,                   % sets the caption-position to bottom
  breaklines=true,                % sets automatic line breaking
  breakatwhitespace=false,        % sets if automatic breaks should only happen at whitespace
  title=\lstname,                 % show the filename of files included with \lstinputlisting;
                                  % also try caption instead of title
  keywordstyle=\color{blue},          % keyword style
  commentstyle=\color{dkgreen},       % comment style
  stringstyle=\color{mauve},         % string literal style
  escapeinside={\%*}{*)},            % if you want to add a comment within your code
  morekeywords={*,...}               % if you want to add more keywords to the set
}

% this is needed for forms and links within the text
\usepackage{hyperref}
\newcommand{\Z}{\mathbb Z}

\newenvironment{query} %% APPROACH 2
 {\begin{minipage}{4cm}\centering}
 {\end{minipage}}

 \newlength\mylength
 \setlength\mylength{\dimexpr.5\columnwidth-2\tabcolsep-0.5\arrayrulewidth\relax}

 \newcommand{\crm}{read\_miss}
 \newcommand{\cwm}{write\_miss}
 \newcommand{\crh}{read\_hit}
 \newcommand{\cwh}{write\_hit}
 \newcommand{\crr}{REQ\_read}
 \newcommand{\crw}{REQ\_write}

 \newcommand{\req}[5]{[#1]: #2 | #3 $\rightarrow$ #4 | $L_#5$}
%
% \newlength\q
% \setlength\q{\dimexpr .5\textwidth -2\tabcolsep}
%
% \newcounter{CicloCounter}
% \begin{table}[h]
%   \centering
%   \caption{Estações de Reserva - Ciclo \theCicloCounter}
%   \begin{tabular}{|p{\q}|p{\q}|p{\q}|p{\q}|p{\q}|p{\q}|p{\q}|p{\q}|}
%     \hline
%     \textbf{ER}     & \textbf{Busy} & \textbf{Op} & \textbf{Qj} & \textbf{Vj} & \textbf{Qk} & \multicolumn{1}{c|}{\textbf{Vk}} & \multicolumn{1}{c|}{\textbf{A}} \\ \hline
%     \textit{add1}   &               &             &             &             &             &                                  &                                 \\ \hline
%     \textit{add2}   &               &             &             &             &             &                                  &                                 \\ \hline
%     \textit{add3}   &               &             &             &             &             &                                  &                                 \\ \hline
%     \textit{mul1}   &               &             &             &             &             &                                  &                                 \\ \hline
%     \textit{mul2}   &               &             &             &             &             &                                  &                                 \\ \hline
%     \textit{load1}  &               &             &             &             &             &                                  &                                 \\ \hline
%     \textit{load2}  &               &             &             &             &             &                                  &                                 \\ \hline
%     \textit{load3}  &               &             &             &             &             &                                  &                                 \\ \hline
%     \textit{load4}  &               &             &             &             &             &                                  &                                 \\ \hline
%     \textit{load5}  &               &             &             &             &             &                                  &                                 \\ \hline
%     \textit{store1} &               &             &             &             &             &                                  &                                 \\ \hline
%     \textit{store2} &               &             &             &             &             &                                  &                                 \\ \hline
%     \textit{store3} &               &             &             &             &             &                                  &                                 \\ \hline
%     \textit{store4} &               &             &             &             &             &                                  &                                 \\ \hline
%     \textit{store5} &               &             &             &             &             &                                  &                                 \\ \hline
%   \end{tabular}
% \end{table}
